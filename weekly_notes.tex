\documentclass[]{article}
% EJ latex definitions and style
\def\tightlist{}
\newcommand*\diff{\mathop{}\!\mathrm{d}}
\newcommand*\red{\color{red}}
\usepackage{lmodern}
\usepackage{amssymb,amsmath}
\usepackage{ifxetex,ifluatex}
\usepackage{fixltx2e} % provides \textsubscript
\ifnum 0\ifxetex 1\fi\ifluatex 1\fi=0 % if pdftex
  \usepackage[T1]{fontenc}
  \usepackage[utf8]{inputenc}
\else % if luatex or xelatex
  \ifxetex
    \usepackage{mathspec}
    \usepackage{xltxtra,xunicode}
  \else
    \usepackage{fontspec}
  \fi
  \defaultfontfeatures{Mapping=tex-text,Scale=MatchLowercase}
  \newcommand{\euro}{€}
\fi
% use upquote if available, for straight quotes in verbatim environments
\IfFileExists{upquote.sty}{\usepackage{upquote}}{}
% use microtype if available
\IfFileExists{microtype.sty}{%
\usepackage{microtype}
\UseMicrotypeSet[protrusion]{basicmath} % disable protrusion for tt fonts
}{}
\usepackage[margin=1in]{geometry}
\ifxetex
  \usepackage[setpagesize=false, % page size defined by xetex
              unicode=false, % unicode breaks when used with xetex
              xetex]{hyperref}
\else
  \usepackage[unicode=true]{hyperref}
\fi
\hypersetup{breaklinks=true,
            bookmarks=true,
            pdfauthor={},
            pdftitle={},
            colorlinks=true,
            citecolor=blue,
            urlcolor=blue,
            linkcolor=magenta,
            pdfborder={0 0 0}}
\urlstyle{same}  % don't use monospace font for urls
\setlength{\parindent}{0pt}
\setlength{\parskip}{6pt plus 2pt minus 1pt}
\setlength{\emergencystretch}{3em}  % prevent overfull lines
\setcounter{secnumdepth}{0}

\date{}

\begin{document}

\subsection{Week 2 Notes (5/29/18)}\label{week-2-notes-52918}

During our meeting:

\begin{itemize}
\tightlist
\item
  Any questions about chapter 1 of Wiggins?
\item
  Review worksheet and answers
\item
  Review/walkthrough code and figures
\item
  Discuss nondimensionalization of gLV equations
\item
  Discuss Lyapunov functions for dynamical systems
\end{itemize}

For next week:

\begin{itemize}
\tightlist
\item
  Explicitly compute the eigenvalues of the Jacobian for the 2D gLV
  equations,

  \begin{align} \begin{split}
      \frac{\text{d}x_a}{\text{d}t} &= x_a (\mu_a - M_{aa} x_a - M_{ab} x_b) \\
      \frac{\text{d}x_b}{\text{d}t} &= x_b (\mu_b - M_{ba} x_a - M_{bb} x_b),
  \end{split} \end{align}

  for the singly-existing steady states \((x^*, \ 0)\) and
  \((0, \ y^*)\), where you should compute \(x^*\) and \(y^*\) by hand
  (feel free to use symbolic computing software to check your answers).
  Compute the stability of these states for the following two parameter
  combinations: (\(\mu_a = \mu_b = 1\), \(M_{aa} = M_{bb} = 1\), and
  \(M_{ab} = M_{ba} = .5\)), and (\(\mu_a = \mu_b = 1\),
  \(M_{aa} = M_{bb} = 1\), and \(M_{ab} = M_{ba} = 1.5\)). Compare these
  results with numerical simulations. In your simulations, use
  \texttt{plt.subplot} in order to make a 2x2 grid of subfigures. Make
  the top row correspond to simulations that start at an initial
  condition of (.1, .9) and the bottom row correspond to an initial
  condition of (.5, .5), and make the left and right columns correspond
  to the simulations that use the first and second parameter sets. Make
  the title of each subfigure indicate which IC and which parameter set
  is used. Do the numerical results agree with your analytic results?
\item
  Read chapter 2 of Wiggins
\item
  Show that 2D gLV equations can be nondimensionalized so that

  \begin{align} \begin{split}
      \frac{\text{d}x_a}{\text{d}t} &= x_a (\mu_a - x_a - M_{ab} x_b) \\
      \frac{\text{d}x_b}{\text{d}t} &= x_b (\mu_b - M_{ba}x_a - x_b),
  \end{split} \end{align}

  and that you may further nondimensionalize time by setting
  \(\mu_a \to 1\).
\item
  Show that the Lyapunov function given by Tang, Yuan and Ma in Phys.
  Rev.~E 2013 satisfies the Lyapunov conditions given in chapter 2 of
  Wiggins. In our notation, the Lyapunov function they provide is

  \begin{equation}
  \begin{split}
    V(x_a, x_b) &= M_{ba} x_a^2/2 + M_{ab} x_b^2/2 \\ 
    &\quad - M_{ba} \mu_a x_a - M_{ab} \mu_b x_b + M_{ab} M_{ba} x_a x_b
  \end{split}
  \end{equation}

  Some hints: 1) remember that \(\hat{x}_a\) and \(\hat{x}_b\) are
  ``directions'' (like \(\hat{x}\) and \(\hat{y}\)) in a 2-dimensional
  space, 2) remember that \(\dot{V} = \nabla V \cdot \dot{\textbf{x}}\),
  where \(\dot{\textbf{x}}\) is the vector form of the dynamical system,
  and 3) assume \(M_{ab} > 0\) and \(M_{ba} >  0\) (this condition must
  be satisfied in order for there to be two stable steady states). Note
  that this equation satifies the Lyapunov conditions except for the
  fact that \(V(\bar{x}) = 0\) for two different \(\bar{x}\),
  corresponding to the two stable steady states. For this reason, this
  is called a \textit{split Lyapunov function}.
\item
  Generalize your code for solving the 2-dimensional gLV equations so
  that it can solve N-dimensional gLV equations. Hint: consider what the
  commands \texttt{np.dot(np.diag(mu), Y)} and
  \texttt{np.dot(np.diag(np.dot(M, Y)), Y)} do, and compare them to the
  N-dimensional gLV equations. Test your code out using the parameters
  from Stein et al., Plos Comp Biol 2013--- I have provided code that
  imports these parameters and turns them into numpy arrays for you. For
  usage, look at \texttt{example\_import\_data.py} and for the
  implementation itself look at \texttt{barebones\_CDI.py}. Compare the
  parameter values from the python code with the Stein paper and ensure
  they agree. This code also imports experimental initial conditions
  from the Stein paper (e.g. \texttt{ic4} in the code). Don't plot your
  output, but try starting from different initial conditions (0-8 are
  allowed; 4 is given as an example in
  \texttt{example\_import\_data.py}) and compare your obtained steady
  states (\texttt{y[:, -1]}) with Table B in
  \texttt{S1\_Appendix\_revision.pdf}
\end{itemize}

\end{document}
